\documentclass[a4paper, 11pt]{article}
\usepackage{graphicx}
\usepackage{amsmath}
\usepackage[pdftex]{hyperref}
\usepackage{algorithm}
\usepackage[noend]{algpseudocode}

\algnewcommand{\Initialize}[1]{%
	\State \textbf{Initialize:}
	\Statex \hspace*{\algorithmicindent}\parbox[t]{.8\linewidth}{\raggedright #1}
}

% Lengths and indenting
\setlength{\textwidth}{16.5cm}
\setlength{\marginparwidth}{1.5cm}
\setlength{\parindent}{0cm}
\setlength{\parskip}{0.15cm}
\setlength{\textheight}{22cm}
\setlength{\oddsidemargin}{0cm}
\setlength{\evensidemargin}{\oddsidemargin}
\setlength{\topmargin}{0cm}
\setlength{\headheight}{0cm}
\setlength{\headsep}{0cm}

\renewcommand{\familydefault}{\sfdefault}

\title{Data Mining: Learning from Large Data Sets - Fall Semester 2015}
\author{mmarti@student.ethz.ch\\ trubeli@student.ethz.ch\\}
\date{\today}

\begin{document}
\maketitle

\section*{Approximate near-duplicate search using Locality Sensitive Hashing} 
In this project we used linear hashing to approximate the similarity between videos, represented by a list of shingles. The first step in our solution was to produce a signature matrix. This matrix is obtained by using a min hash algorithm on each list of shingles. For every $i^{th}$ shingle in a video we pick two random numbers $a_{i}$ and $b_{i}$ which are coprime. The procedure for computing the signature matrix is described as follow:
\vspace{10mm}
\begin{algorithm}
	\caption{Min Hash Algorithm}\label{euclid}
	\begin{algorithmic}[1]
		\Procedure{MinHash}{$N,K$}\Comment{K hash fonction applied on N shingles}
		\Initialize{\strut$w_l \gets \infty$, $l=1,\ldots,k$}	
		\For{$i = 1 $ \textbf{to} $n$}
			\For{$j = 1 $ \textbf{to} $k$}
			\If{$h_{j}(n_{i}) < w_{j}$}
				\State{$w_{j} \gets h_{j}(n_{i})$}
			\EndIf
			\EndFor
		\EndFor
	
		\EndProcedure
	\end{algorithmic}
\end{algorithm}

\end{document} 
